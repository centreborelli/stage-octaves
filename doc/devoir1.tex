\documentclass[a4paper,11pt]{article}
\usepackage[utf8]{inputenc}
\usepackage[french]{babel}
\usepackage{amsmath}
\usepackage[osf,sups]{Baskervaldx} % lining figures
\usepackage[bigdelims,cmintegrals,vvarbb,baskervaldx,frenchmath]{newtxmath} % math font
%\usepackage[cal=boondoxo]{mathalfa} % mathcal from STIX, unslanted a bit
%\usepackage{float}
\usepackage{graphicx}
\usepackage{url,hyperref}
%\usepackage{color}

\setlength{\parindent}{0em}
\setlength{\parskip}{1em}
%\addtolength{\hoffset}{-3.5em}
%\addtolength{\textwidth}{7em}
%\addtolength{\voffset}{-5em}
%\addtolength{\textheight}{10em}

%\addtolength{\voffset}{-6em}
%\addtolength{\textheight}{12em}

\begin{document}
\thispagestyle{empty}

\begin{center}
	\Large Vibrations d'un système discret
\end{center}

On considère les trois matrices suivantes:
\[
	D_3=\begin{pmatrix}
		-2 & 1 & 0 \\
		1 & -2 & 1 \\
		0 & 1 & -2 \\
	\end{pmatrix}
	\qquad
	N_3=\begin{pmatrix}
		-1 & 1 & 0 \\
		1 & -2 & 1 \\
		0 & 1 & -1 \\
	\end{pmatrix}
	\qquad
	P_3=\begin{pmatrix}
		-2 & 1 & 1 \\
		1 & -2 & 1 \\
		1 & 1 & -2 \\
	\end{pmatrix}
\]

{\bf Exercice 1.}
Calculez explicitement (à la main) les valeurs et vecteurs propres des
matrices~$X_3$, pour~$X=D,N,P$.

{\sl Indication: les vecteurs propres sont des évaluations des fonctions
trigonométriques sur des positions régulièrement espacées.}

{\bf Exercice 2.}
Définissez des matrices tridiagonales~$D_n$, $N_n$ et $P_n$, de taille~$n\times
n$ pour~$n\ge 3$, qui généralisent le cas particulier~$n=3$.

{\bf Exercice 3.}
À la vue de l'exercice~$1$, trouver les valeurs et vecteurs propres des
matrices~$D_n$, ~$N_n$ et~$P_n$ pour~$n\ge 3$.

{\bf Définition.}
Si~$A$ est une matrice positive, on dénote par~$\lambda_k(A)$, $k=1,2,\ldots$
les valeurs propres strictement positifs de~$A$, ordonnées par ordre croissant.

{\bf Exercice 4.}
Vérifiez que les matrices~$-D_n$, $-N_n$ et $-P_n$ sont positives et donnez une
expression fermée pour les~\emph{partiels}
\[
	\mu_k(X_n) := \sqrt{\frac{\lambda_k(-X_n)}{\lambda_1(-X_n)}}
\]
pour~$X=D,N,P$ et~$k\ge 1$.  Démontrez que quand~$n>>k$ on a~$\mu_k\approx k$,
et estimez la qualité de cette approximation (selon la valeur de~$\tfrac kn$).

{\bf Exercice 5.}
Donnez une interprétation physique à ces matrices et à leurs vecteurs propres.

{\bf Exercice 6.}
Donnez une version continue de toutes ces constructions.


\end{document}


% vim:set tw=79 spell spelllang=fr:
